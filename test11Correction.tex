\documentclass[a4paper,12pt]{article}
\usepackage{graphicx}
\usepackage[a4paper, top=0cm, bottom=2cm, left=2cm, right=2cm]{geometry} % Ajuste les marges
\usepackage{xcolor} % Pour ajouter des couleurs
\usepackage{hyperref} % Pour avoir des références colorées si nécessaire
\usepackage{eso-pic} % Pour ajouter des éléments en arrière-plan

\usepackage[french]{babel}
\usepackage[T1]{fontenc}
\usepackage{mathrsfs}
\usepackage{amsmath}
\usepackage{amsfonts}
\usepackage{amssymb}
\usepackage{tkz-tab}

\usepackage{tikz}
\usetikzlibrary{arrows, shapes.geometric, fit}
\newcounter{correction} % Compteur pour les questions

% Définir la commande pour afficher une question numérotée
\newcommand{\question}{%
  \refstepcounter{correction}%
  \textbf{\textcolor{black}{Question \thecorrection (1 point) :}} \ignorespaces
}
% Commande pour ajouter du texte en arrière-plan
\AddToShipoutPicture{
    \AtTextCenter{%
        \makebox[50pt]{\rotatebox{80}{\textcolor[gray]{0.5}{\fontsize{10cm}{10cm}\selectfont PGB}}}
    }
}

\begin{document}
\hrule % Barre horizontale
% En-tête
\begin{center}
    \begin{tabular}{@{} p{5cm} p{5cm} p{5cm} @{}} % 3 colonnes avec largeurs fixées
        Lycée Dindéfélo & \quad\quad Teste 11 & 07 Mars 2024 \\
    \end{tabular}
    \\[-0.01cm] % Ajuster l'espace vertical entre le tableau et la barre
    \hrule % Barre horizontale
\end{center}
\begin{center}
    \textbf{\Large Nombre Complexe} \\[0.2cm]
    \textbf{\large Professeur : M. BA} \\[0.2cm]
    \textbf{Classe : Terminale S2} \\[0.2cm]
    \textbf{\small Durée : 10 minutes} \\[0.2cm]
    \textbf{\small Note :\quad\quad /5}
\end{center}

% Nom de l'élève
\textbf{\small Nom de l'élève :} \underline{\hspace{8cm}} \\[0.5cm]

\begin{enumerate}
\item Pour tout nombre complexe $z \neq -1 + 2i$, on pose $Z = \frac{z - 2 + 4i}{z + 1 - 2i}$.

Déterminons l'ensemble des points $M$ du plan tels que :

		$$Z = \frac{x^2 + y^2 - x + 2y - 10 + i(6x + 3y)}{(x+1)^2 + (y-2)^2}$$


\begin{enumerate}
        \item $|Z| = 1$
        \item $Z$ soit un réel.
\end{enumerate}

\item Pour tout complexe $z \neq i$, on pose
    \(
    U = \frac{z + i}{z - i}.
    \)
    
    Déterminer l’ensemble des points $M$ d’affixe $z$ tels que :
    \begin{enumerate}
        \item $U \in \mathbb{R}^*_{-}$
        \item $U \in i\mathbb{R}$
    \end{enumerate}

\end{enumerate}

\end{document}