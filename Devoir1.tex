\documentclass[a4paper,12pt]{article}
\usepackage{graphicx}
\usepackage[a4paper, top=0cm, bottom=2cm, left=2cm, right=2cm]{geometry} % Ajuste les marges
\usepackage{xcolor} % Pour ajouter des couleurs
\usepackage{hyperref} % Pour avoir des références colorées si nécessaire
\usepackage{eso-pic}         % Pour ajouter des éléments en arrière-plan

\usepackage[french]{babel}
\usepackage[T1]{fontenc}
\usepackage{mathrsfs}
\usepackage{amsmath}
\usepackage{amsfonts}
\usepackage{amssymb}
\usepackage{tkz-tab}

\usepackage{tikz}
\usetikzlibrary{arrows, shapes.geometric, fit}
\newcounter{correction} % Compteur pour les questions

% Définir la commande pour afficher une question numérotée
\newcommand{\question}{%
  \refstepcounter{correction}%
  \textbf{\textcolor{black}{Question \thecorrection : 2 pts}} \ignorespaces
}
% Commande pour ajouter du texte en arrière-plan
\AddToShipoutPicture{
    \AtTextCenter{%
        \makebox[0pt]{\rotatebox{80}{\textcolor[gray]{0.9}{\fontsize{10cm}{10cm}\selectfont Pathé Gobel BA}}}
    }
}
\begin{document}
\hrule % Barre horizontale
% En-tête
\begin{center}
    \begin{tabular}{@{} p{5cm} p{5cm} p{5cm} @{}} % 3 colonnes avec largeurs fixées
        Lycée Dindéfélo & \quad\quad\quad Test 4 & 4 Décembre 2024 \\
    \end{tabular}
    \\[-0.01cm] % Ajuster l'espace vertical entre le tableau et la barre
    \hrule % Barre horizontale
\end{center}
\begin{center}
    \textbf{\Large Equations Paramétriques} \\[0.2cm]
    \textbf{\large Professeur : M. BA} \\[0.2cm]
    \textbf{Classe : $1^{er}$ S2} \\[0.2cm]
    \textbf{\small Durée : 10 minutes} \\[0.2cm]
    \textbf{\small Note :\quad\quad /5}
\end{center}

\textbf{Problème : 12,75 points}

\textbf{Partie A :}

Soit \( f \) la fonction définie par :
\[
f(x) = x - 2 - \sqrt{x^2 - 2x}.
\]


\textbf{1. Déterminer \( D_f \).} \hspace{1cm} (0,5 pt)

\begin{enumerate}
    \item[a)] Calculer \( \lim_{x \to +\infty} f(x) \) et \( \lim_{x \to -\infty} f(x) \). \hspace{1cm} (0,25 pt), (0,5 pt)
    \item[b)] Étudier la branche infinie de la courbe \( (C_f) \) au voisinage de \( -\infty \). \hspace{1cm} (1 pt)
    \item[c)] Étudier la branche infinie de la courbe \( (C_f) \) au voisinage de \( +\infty \). \hspace{1cm} (1 pt)
\end{enumerate}

2. Étudier la dérivabilité de la fonction \( f \) à droite de 2 et à gauche de 0, puis interpréter géométriquement les résultats obtenus. \hspace{1cm} (2 pts)

\begin{enumerate}
    \item[a)] Justifier la dérivabilité de la fonction sur \( ]-\infty, 0] \cup ]2, +\infty[ \), puis montrer que pour tout \( x \in ]-\infty, 0[ \cup ]2, +\infty[ \) :
    \[
    f'(x) = \frac{\sqrt{x^2 - 2x} - (x - 1)}{\sqrt{x^2 - 2x}}.
    \]
    \hspace{1cm} (1,5 pt)
    
    \item[b)] Montrer que : \( \forall x \in ]-\infty, 0], f'(x) > 0 \) et \( \forall x \in ]2, +\infty[, f'(x) < 0 \). \hspace{1cm} (1 pt)
    
    \item[c)] Dresser le tableau de variations de la fonction \( f \). \hspace{1cm} (1,25 pt)
\end{enumerate}

3. Tracer la courbe \( (C_f) \) dans un repère orthonormé \( (O, \vec{i}, \vec{j}) \). \hspace{1cm} (1,25 pt)

\textbf{Partie B :}

On considère la fonction \( g \) la restriction de la fonction \( f \) sur \( [2, +\infty[ \) :
\[
g(x) = f(x), \quad x \geq 2.
\]

\begin{enumerate}
    \item[a)] Montrer que \( g \) admet une fonction réciproque \( g^{-1} \) définie sur un intervalle \( J \) qu'on déterminera. \hspace{1cm} (0,5 pt)
    
    \item[b)] Calculer \( g^{-1}(2 - 2\sqrt{2}) \). (On donne : \( g(4) = 2 - 2\sqrt{2} \)). \hspace{1cm} \textbf{(0,75 pt)}
    
    \item[c)] Déterminer \( g^{-1}(x) \) pour tout \( x \in J \). \hspace{3cm} \textbf{(0,5 pt)}
    
    \item[d)] Tracer la courbe \( (C_{g^{-1}}) \) dans le même repère orthonormé \( (O, \vec{i}, \vec{j}) \). \hspace{2cm} \textbf{(0,75 pt)}
\end{enumerate}

\end{document}