\documentclass[a4paper,12pt]{article}
\usepackage{amsmath}
\usepackage{geometry}
\geometry{top=2cm, bottom=2cm, left=2cm, right=2cm}

\begin{document}

\title{Évaluation 1 - Chapitre : Composition des applications}
\author{}
\date{}
\maketitle

\noindent \textbf{Lycée Jatakunda} \hfill \textbf{Professeur : M. Janta} \\
\noindent \textbf{Classe : TL2} \hfill \textbf{Durée : 1 heure} \\
\noindent \textbf{Note totale : 20 points} \\

\vspace{0.5cm}

\section*{Exercice 1 : QCM (5 points)}
Choisissez la bonne réponse parmi les propositions suivantes :

\begin{enumerate}
    \item Une application $f$ est définie par $f : \mathbb{R} \to \mathbb{R}$ avec $f(x) = \frac{x + 2}{x - 3}$. Le domaine de définition de $f$ est :
    \begin{itemize}
        \item[a)] $\mathbb{R} \setminus \{-3\}$
        \item[b)] $\mathbb{R} \setminus \{3\}$
        \item[c)] $\mathbb{R} \setminus \{0\}$
        \item[d)] $\mathbb{R} \setminus \{-2\}$
    \end{itemize}
    
    \item Si $h(x) = \frac{x^2 - 1}{x + 1}$, alors le domaine de définition de $h$ est :
    \begin{itemize}
        \item[a)] $\mathbb{R} \setminus \{-1\}$
        \item[b)] $\mathbb{R} \setminus \{1\}$
        \item[c)] $\mathbb{R}$
        \item[d)] $\mathbb{R} \setminus \{0\}$
    \end{itemize}
    
    \item La fonction composée $f \circ h$ est définie si :
    \begin{itemize}
        \item[a)] $x$ appartient à $D_h$ tel que $h(x)$ appartient à $D_f$
        \item[b)] $f(x)$ appartient à $D_h$
        \item[c)] $f$ est une fonction linéaire
        \item[d)] $h$ est une application croissante
    \end{itemize}
    
    \item Une application $f$ de $A$ vers $B$ est surjective si :
    \begin{itemize}
        \item[a)] Tout élément de $B$ a un antécédent dans $A$
        \item[b)] Tout élément de $A$ a une image dans $B$
        \item[c)] Le domaine de $f$ est égal à l’ensemble de départ
        \item[d)] Aucun élément de $B$ n’a d’antécédent dans $A$
    \end{itemize}
    
    \item Si $g(x) = x^3 + x$, alors l’image de $1$ par $g$ est :
    \begin{itemize}
        \item[a)] $0$
        \item[b)] $2$
        \item[c)] $3$
        \item[d)] $1$
    \end{itemize}
\end{enumerate}

\section*{Exercice 2 : Complétez les définitions (4 points)}
Remplissez les espaces vides avec les mots appropriés. Utilisez autant de détails que possible :

\vspace{0.5cm}
1. Une fonction $g$ est une application si \underline{\hspace{15cm}}  
\vspace{0.5cm}

2. L'image d'un élément $a$ par une fonction $g$ est \underline{\hspace{15cm}}  
\vspace{0.5cm}

3. Le domaine de définition de $h$ est l'ensemble des \underline{\hspace{15cm}}  
\vspace{0.5cm}

4. Une fonction composée est \underline{\hspace{15cm}}  
\vspace{0.5cm}

\section*{Exercice 3 : Déterminations et calculs (6 points)}
Soit $f$ et $g$ définies par : 
\[
f(x) = \frac{x}{x + 2}, \quad g(x) = x^2 + 1
\]

\begin{enumerate}
    \item Déterminer le domaine de définition de $f$, $g$, et des fonctions composées $g \circ f$ et $f \circ g$. 
    \begin{itemize}
        \item Le domaine de $g \circ f$ est l'ensemble des $x$ appartenant à $D_f$ tel que $f(x)$ appartient à $D_g$.
        \item Le domaine de $f \circ g$ est l'ensemble des $x$ appartenant à $D_g$ tel que $g(x)$ appartient à $D_f$.
    \end{itemize}
    
    \item Calculer $f(g(3))$ et $g(f(3))$.
    
    \item Comparer les résultats obtenus pour $g \circ f$ et $f \circ g$.
\end{enumerate}

\section*{Barème proposé :}
\begin{itemize}
    \item \textbf{Exercice 1 (QCM)} : 5 points (1 point par question)
    \item \textbf{Exercice 2 (Complétez les définitions)} : 4 points (1 point par espace complété)
    \item \textbf{Exercice 3 (Déterminations et calculs)} : 6 points (2 points par question)
\end{itemize}

\end{document}