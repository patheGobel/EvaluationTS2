\documentclass[12pt,a4paper]{article}
\usepackage[utf8]{inputenc}
\usepackage[T1]{fontenc}
\usepackage{amsmath,amssymb}
\usepackage{geometry}
\geometry{top=2cm, bottom=2cm, left=2cm, right=2cm}

\begin{document}

\section*{PROBLEME : 10 points}

\subsection*{PARTIE A : 2,5 points}

Soit $u$ la fonction définie sur $\mathbb{R}$ par $u(x) = x^3 + 3x - 1$,

\begin{enumerate}
    \item[a)] Dresser le tableau de variation de $u$. \hfill \textbf{0,5 pt}
    \item[b)] Montrer que l'équation $u(x) = 0$ admet une solution unique $\alpha$ et que $0 < \alpha < 1$. \hfill \textbf{1 pt}
    \item[c)] Donner un encadrement de $\alpha$ à $10^{-1}$ près. \hfill \textbf{0,5 pt}
    \item[d)] En déduire le signe de $u(x) - 1$ sur $\mathbb{R}$. \hfill \textbf{0,5 pt}
\end{enumerate}

\subsection*{PARTIE B : 7,5 points}

Soit $f$ la fonction définie par :
\[
f(x) = 
\begin{cases} 
    \frac{2(x^3 + 1)}{x^2 + 1} & \text{si } x < 1, \\
    f(x) = 1 + \sqrt{2x - 1} & \text{si } x \geq 1.
\end{cases}
\]

On désigne par $(\mathcal{C}_f)$ la courbe de $f$ dans le plan muni d’un repère orthonormé direct 
$\left(O, \vec{i}, \vec{j}\right)$ d’unité graphique 2 cm.

\begin{enumerate}
    \item Montrer que $f$ est définie sur $\mathbb{R}$. \hfill \textbf{0,5 pt}
    \item Étudier la continuité et la dérivabilité de $f$ en 1. \hfill \textbf{1 pt}
    \item Écrire une équation de la tangente $(T)$ à $(\mathcal{C}_f)$ au point $A(1,2)$. \hfill \textbf{0,25 pt}
    \item Étudier la position relative de $(\mathcal{C}_f)$ par rapport à $(T)$. \hfill \textbf{0,25 pt}
		\item
		\begin{enumerate}
		    \item Montrer que pour tout $x \in ]-\infty;1]$, $f'(x) = \frac{2x\big[u(x) - 1\big]}{(x^2 + 1)^2}$. \hfill \textbf{2,5 pt}
    \item Calculer $f'(x)$ pour $x \in [1,+\infty[$. \hfill \textbf{0,5 pt}
    \item Étudier le signe de $f'(x)$ sur $\mathbb{R}$. \hfill \textbf{0,5 pt}
    \item Dresser le tableau de variation de $f$. \hfill \textbf{1 pt}
		\end{enumerate}
		\item 
		\begin{enumerate}
    \item Étudier les branches infinies de $(\mathcal{C}_f)$. \hfill \textbf{1,25 pt}
    \item Tracer $(\mathcal{C}_f)$. (On prendra $\alpha \approx 0,6$). \hfill \textbf{0,75 pt}
		\end{enumerate}
		
		\item  Soit $g$ la restriction de $f$ sur $[1, +\infty[$.
\begin{enumerate}
    \item Montrer que $g$ réalise une bijection de $[1, +\infty[$ vers un intervalle $J$ à déterminer. \hfill \textbf{0,25 pt}
    \item Soit $g^{-1}$ la bijection réciproque de $g$. Étudier la dérivabilité de $g^{-1}$ et donner son sens de variation. \hfill \textbf{0,5 pt}
    \item Tracer la courbe de $g^{-1}$ dans le même repère. \hfill \textbf{0,5 pt}
\end{enumerate}
\end{enumerate}

\end{document}
