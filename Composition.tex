\documentclass[12pt,a4paper]{article}
\usepackage{amsmath,amssymb,mathrsfs,tikz,times,pifont}
\usepackage{enumitem}
\newcommand\circitem[1]{%
\tikz[baseline=(char.base)]{
\node[circle,draw=gray, fill=red!55,
minimum size=1.2em,inner sep=0] (char) {#1};}}
\newcommand\boxitem[1]{%
\tikz[baseline=(char.base)]{
\node[fill=cyan,
minimum size=1.2em,inner sep=0] (char) {#1};}}
\setlist[enumerate,1]{label=\protect\circitem{\arabic*}}
\setlist[enumerate,2]{label=\protect\boxitem{\alph*}}
%%%::::::by chnini ameur :::::::%%%
\everymath{\displaystyle}
\usepackage[left=1cm,right=1cm,top=1cm,bottom=1.7cm]{geometry}
\usepackage{array,multirow}
\usepackage[most]{tcolorbox}
\usepackage{varwidth}
\tcbuselibrary{skins,hooks}
\usetikzlibrary{patterns}
%%%::::::by chnini ameur :::::::%%%
\newtcolorbox{exa}[2][]{enhanced,breakable,before skip=2mm,after skip=5mm,
colback=yellow!20!white,colframe=black!20!blue,boxrule=0.5mm,
attach boxed title to top left ={xshift=0.6cm,yshift*=1mm-\tcboxedtitleheight},
fonttitle=\bfseries,
title={#2},#1,
% varwidth boxed title*=-3cm,
boxed title style={frame code={
\path[fill=tcbcolback!30!black]
([yshift=-1mm,xshift=-1mm]frame.north west)
arc[start angle=0,end angle=180,radius=1mm]
([yshift=-1mm,xshift=1mm]frame.north east)
arc[start angle=180,end angle=0,radius=1mm];
\path[left color=tcbcolback!60!black,right color = tcbcolback!60!black,
middle color = tcbcolback!80!black]
([xshift=-2mm]frame.north west) -- ([xshift=2mm]frame.north east)
[rounded corners=1mm]-- ([xshift=1mm,yshift=-1mm]frame.north east)
-- (frame.south east) -- (frame.south west)
-- ([xshift=-1mm,yshift=-1mm]frame.north west)
[sharp corners]-- cycle;
},interior engine=empty,
},interior style={top color=yellow!5}}
%%%%%%%%%%%%%%%%%%%%%%%

\usepackage{fancyhdr}
\usepackage{eso-pic}         % Pour ajouter des éléments en arrière-plan
% Commande pour ajouter du texte en arrière-plan
\AddToShipoutPicture{
    \AtTextCenter{%
        \makebox[0pt]{\rotatebox{80}{\textcolor[gray]{0.7}{\fontsize{5cm}{5cm}\selectfont PGB}}}
    }
}
\usepackage{lastpage}
\fancyhf{}
\pagestyle{fancy}
\renewcommand{\footrulewidth}{1pt}
\renewcommand{\headrulewidth}{0pt}
\renewcommand{\footruleskip}{10pt}
\fancyfoot[R]{
\color{blue}\ding{45}\ \textbf{2025}
}
\fancyfoot[L]{
\color{blue}\ding{45}\ \textbf{Prof:M. BA}
}
\cfoot{\bf
\thepage /
\pageref{LastPage}}
\begin{document}
\renewcommand{\arraystretch}{1.5}
\renewcommand{\arrayrulewidth}{1.2pt}
\begin{tikzpicture}[overlay,remember picture]
\node[draw=blue,line width=1.2pt,fill=purple,text=blue,inner sep=3mm,rounded corners,pattern=dots]at ([yshift=-2.5cm]current page.north) {\begingroup\setlength{\fboxsep}{0pt}\colorbox{white}{\begin{tabular}{|*1{>{\centering \arraybackslash}p{0.28\textwidth}} |*2{>{\centering \arraybackslash}p{0.2\textwidth}|} *1{>{\centering \arraybackslash}p{0.19\textwidth}|} }
\hline
\multicolumn{3}{|c|}{$\diamond$$\diamond$$\diamond$\ \textbf{Lycée de Dindéfélo}\ $\diamond$$\diamond$$\diamond$ }& \textbf{A.S. : 2024/2025} \\ \hline
\textbf{Matière: Mathématiques}& \textbf{Niveau : T}\textbf{S2} &\textbf{Date: 30/01/2024} & \textbf{Durée : 4 heures} \\ \hline
\multicolumn{4}{|c|}{\parbox[c]{10cm}{\begin{center}
\textbf{{\Large\sffamily composition n$ ^{\circ} $ 1 Du 1$ ^\text{\bf er} $ Semestre}}
\end{center}}} \\ \hline
\end{tabular}}\endgroup};
\end{tikzpicture}
\vspace{3cm}

\section*{\underline{Exercice 1 :} $1,5$ points}

\begin{enumerate}

\item Déterminons les racines cubiques de l'unité.

Résolvons pour cela $z^3 = 1$, $z \in \mathbb{C}$ :

\[
z^3 = 1 \implies z^3 = e^{i(0+2k\pi)}, \quad k \in \mathbb{Z}.
\]

\[
z = e^{i\frac{(0+2k\pi)}{3}}, \quad k = 0, 1, 2.
\]

- Pour $k = 0$, $z = e^{i0} = 1$, donc $z_0 = 1$.

- Pour $k = 1$, $z = e^{i\frac{2\pi}{3}} = \cos\frac{2\pi}{3} + i\sin\frac{2\pi}{3} = -\frac{1}{2} + i\frac{\sqrt{3}}{2}$, donc $z_1 = -\frac{1}{2} + i\frac{\sqrt{3}}{2}$.

- Pour $k = 2$, $z = e^{i\frac{4\pi}{3}} = \cos\frac{4\pi}{3} + i\sin\frac{4\pi}{3} = -\frac{1}{2} - i\frac{\sqrt{3}}{2}$, donc $z_2 = -\frac{1}{2} - i\frac{\sqrt{3}}{2}$.

Ainsi, l'ensemble des solutions est :

\[
S_{\mathbb{C}} = \left\{ 1, -\frac{1}{2} + i\frac{\sqrt{3}}{2}, -\frac{1}{2} - i\frac{\sqrt{3}}{2} \right\}.
\]

\item Interprétation

\begin{itemize}
    \item $\text{arg}(z_B - z_A) = (\overrightarrow{OI}, \overrightarrow{AB}) \, [2\pi]$
    
    Géométriquement, $\text{arg}(z_B - z_A)$ est l'angle $\widehat{(\overrightarrow{OI}, \overrightarrow{AB})}$ formé par le vecteur $\overrightarrow{AB}$ avec l'axe réel du plan complexe $(O, \overrightarrow{OI})$.

    \item $\text{arg}\left(\frac{z_B - z_C}{z_B - z_A}\right) = (\overrightarrow{AB}, \overrightarrow{CD}) \, [2\pi]$
    
    Géométriquement, $\text{arg}\left(\frac{z_B - z_C}{z_B - z_A}\right)$ est l'angle $\widehat{(\overrightarrow{AB}, \overrightarrow{CD})}$ que forment les vecteurs $\overrightarrow{AB}$ et $\overrightarrow{CD}$.
\end{itemize}

\item $f(x) = \sqrt{x}$, appliquons l'IAF

Appliquons l'IAF (Inégalité des Accroissements Finis) sur $[l, l+1]$.

\textbf{TI-AF 1}

Si $f$ est continue sur $[a, b]$ et dérivable sur $]a, b[$, avec 
\[
\exists (m, M) \in \mathbb{R}, \forall x \in [a, b], \quad \text{tel que } m \leq f'(x) \leq M,
\]
alors :
\[
m(b-a) \leq f(b) - f(a) \leq M(b-a).
\]

\textbf{TI-AF 2}

Si $f$ est continue sur $[a, b]$ et dérivable sur $]a, b[$, avec 
\[
\exists m \in \mathbb{R}_{+}, \forall x \in [a, b], \quad \text{tel que } |f'(x)| \leq m,
\]
alors :
\[
|f(b) - f(a)| \leq m|b-a|.
\]

\newpage
\textbf{Applications} $t > 0$

\begin{itemize}
    \item $f$ est continue sur $[t, t+1]$.
    \item $f$ est dérivable sur $]t, t+1[$ et pour tout,
\[
    \forall x \in ]t, t+1[, \quad f'(x) = \frac{1}{2\sqrt{x}}.
\]
\end{itemize}

Comme $x \in ]t, t+1[$, alors $t < x < t+1$. Ainsi, 
\[
\frac{1}{2\sqrt{t+1}} < \frac{1}{2\sqrt{x}} < \frac{1}{2\sqrt{t}}.
\]

D'où :
\[
\frac{1}{2\sqrt{t+1}} < f'(x) < \frac{1}{2\sqrt{t}}.
\]

En appliquant l'inégalité des accroissements finis (IAF), nous avons :
\[
\frac{1}{2\sqrt{t+1}}(t+1-t) < f(t+1) - f(t) < \frac{1}{2\sqrt{t}}(t+1-t).
\]

Cela donne :
\[
\frac{1}{2\sqrt{t+1}} < f(t+1) - f(t) < \frac{1}{2\sqrt{t}}.
\]

\end{enumerate}

\section*{\underline{Exercice 2 :} 8,5 points}

\subsection*{ Partie A}
\begin{enumerate}
\item Montrons que $P$ admet une unique racine réelle.

Soit $z_0 = a \in \mathbb{R}$ tel que $P(z_0) = 0$.

\[
P(z) = z^3 + (-5 + 2i)z^2 + (7 - 7i)z - 2 + 6i
\]

\[
P(z_0) = 0 \implies a^3 + (-5 + 2i)a^2 + (7 - 7i)a - 2 + 6i = 0
\]

\[
\implies a^3 - 5a^2 + 7a - 2 + i\big(2a^2 - 7a + 6\big) = 0
\]

En séparant les parties réelle et imaginaire :
\[
\begin{cases}
a^3 - 5a^2 + 7a - 2 = 0 \quad (1) \\
2a^2 - 7a + 6 = 0 \quad (2)
\end{cases}
\]

L'équation (2) est plus facile à résoudre.

Résolution de (2)

\[
2a^2 - 7a + 6 = 0
\]

Calculons le discriminant :
\[
\Delta = 7^2 - 4 \times 2 \times 6 = 49 - 48 = 1.
\]

Les solutions sont :
\[
a_1 = \frac{-(-7) - \sqrt{\Delta}}{2 \times 2} = \frac{7 - 1}{4} = \frac{6}{4} = \frac{3}{2},
\]
\[
a_2 = \frac{-(-7) + \sqrt{\Delta}}{2 \times 2} = \frac{7 + 1}{4} = \frac{8}{4} = 2.
\]

La racine de (2) qui vérifie (1) est la racine de $P(z)$.

Vérification pour $a = \frac{3}{2}$

\underline{Substituons $a = \frac{3}{2}$ dans (1) :}
\begin{align*}
a^3 - 5a^2 + 7a - 2 &= \left(\frac{3}{2}\right)^3 - 5\left(\frac{3}{2}\right)^2 + 7\left(\frac{3}{2}\right) - 2\\
										&=\frac{27}{8} - 5 \times \frac{9}{4} + \frac{21}{2} - 2\\
										&=\frac{27}{8} - \frac{45}{8} + \frac{84}{8} - \frac{16}{8}\\
										&=\frac{27 - 45 + 84 - 16}{8}\\
										&= \frac{111 - 106}{8} = \frac{5}{8} \neq 0
\end{align*}

Donc, $a = \frac{3}{2}$ n'est pas solution.

\underline{Vérification pour $a = 2$}

Substituons $a = 2$ dans (1) :

\begin{align*}
a^3 - 5a^2 + 7a - 2 &= 2^3 - 5(2^2) + 7(2) - 2\\
										&=8 - 20 + 14 - 2\\
										&=0
\end{align*}

Donc, $a = 2$ est une solution.

\[
z_0 = 2.
\]

\item Factorisons $P(z)$

Utilisons la méthode de la division synthétique pour factoriser $P(z)$. Le tableau est donné ci-dessous :

\[
\begin{array}{|c|c|c|c|c|}
\hline
\text{ } & 1 & -5+2i & 7-7i & 2+6i \\ 
\hline
2 & \downarrow & 2 & -6+i & 2-6i \\ 
\hline
\text{ } & 1 & -3+2i & 1-3i & 0 \\ 
\hline
\end{array}
\]

Ainsi, nous avons :
\[
P(z) = (z - 2)\big(z^2 + (-3+2i)z + 1-3i\big).
\]

\item Résolvons $P(z) = 0$

\[
P(z) = 0 \implies z = 2 \quad \text{ou} \quad z^2 + (-3+2i)z + 1-3i = 0.
\]

\begin{align*}
\Delta &= (-3+2i)^2 - 4(1)(1-3i)\\
 			 &= 9 - 12i - 4 + 12i\\
	     &= 5
\end{align*}

\[
z_{1} = \frac{3-\sqrt{5}}{2}-i, z_{2} = \frac{3+\sqrt{5}}{2}-i
\]

\[
S = \left\{ 2, \frac{3-\sqrt{5}}{2}-i, \frac{3+\sqrt{5}}{2}-i \right\}.
\]

\end{enumerate}
\subsection*{ Partie B}

\begin{enumerate}
\item Représentation

\begin{center}
\begin{tikzpicture}[scale=2]

    % Axes avec origine décalée à gauche
    \draw[->] (-1,0) -- (3.5,0) node[below] {\( \text{Re} \)};
    \draw[->] (0,-1.5) -- (0,2.5) node[left] {\( \text{Im} \)};
    
    % Cercle unité pour illustration de l'angle
    \draw[dashed, gray] (0,0) circle(1);

    % Points
    \filldraw[red] (2,-1) circle(2pt) node[below right] {\( A(2,-1) \)};
    \filldraw[red] (2,0) circle(2pt) node[above right] {\( B(2) \)};
    \filldraw[red] (1,-1) circle(2pt) node[below left] {\( C(1,-1) \)};
    \filldraw[red] (1,0) circle(2pt) node[above left] {\( D(1) \)};
    
    % Lignes de connexion
    \draw[thick, dashed, blue] (2,-1) -- (2,0) -- (1,0) -- (1,-1) -- cycle;
    
    % Angle de π/3
    \draw[thick] (0.7,0) arc[start angle=0, end angle=60, radius=0.7];
    \node at (0.9,0.3) {\(\frac{\pi}{3}\)};
    
    % Vecteur illustrant un nombre complexe sur le cercle
    \draw[thick, ->] (0,0) -- (cos{60},sin{60});
    
\end{tikzpicture}
\end{center}

\item Écriture exponentielle de $z_C$ et $z_E$

\underline{Pour $z_C$ :}

\begin{align*}
z_C &= 1 - i\\
		&= \sqrt{2} \left(\cos\left(-\frac{\pi}{4}\right) + i\sin\left(-\frac{\pi}{4}\right)\right)\\
		&= \sqrt{2} e^{-i\pi/4}\\
\end{align*}
 \[ \boxed{ z_C=\sqrt{2} e^{-i\pi/4}} \]
\underline{Pour $z_E$ :}

\begin{align*}
z_E &= 1 + i\sqrt{3}\\
	  &= 2 \left(\frac{1}{2} + i\frac{\sqrt{3}}{2}\right)\\
		&= 2 \left(\cos\left(\frac{\pi}{3}\right) + i\sin\left(\frac{\pi}{3}\right)\right)\\
		&= 2 e^{i\pi/3}
\end{align*}

Donc :
\[ \boxed{z_E = 2 e^{i\pi/3}} \]

\item 3. Plaçons exactement le point $E$

\[
z_E = 2 e^{i\pi/3}.
\]

\item Écriture algébrique de $z_E^8$

\begin{align*}
z_E = 2 e^{i\pi/3} &\implies z_E^8 = 2^8 e^{8i\pi/3}\\
                   &\implies z_E^8=2^8 \left[\cos\left(\frac{2\times 3\pi}{3}+\frac{2\pi}{3}\right) + i\sin\left(\frac{2\times 3\pi}{3}+\frac{2\pi}{3}\right)\right]\\
										&\implies z_E^8=\left[\cos\left(2\pi+\frac{2\pi}{3}\right) + i\sin\left(2\pi+\frac{2\pi}{3}\right)\right]\\
										&\implies z_E^8 = 2^8 \left(-\frac{1}{2} + i\frac{\sqrt{3}}{2}\right)
\end{align*}

\[
\boxed{z_E^8 = 2^8 \left(-\frac{1}{2} + i\frac{\sqrt{3}}{2}\right)}.
\]

\item 5. Calculons les affixes de $I$ et $J$

On a : $ I = \text{bar}\{(A,1); (C,-2)\} $
\begin{align*}
I = \text{bar}\{(A,1); (C,-2)\} &\implies \overrightarrow{AI} = \frac{-2}{1-2} \overrightarrow{AC}\\
                                &\implies \overrightarrow{AI} = 2 \overrightarrow{AC}\\
                                &\implies Z_I - Z_A = 2 (Z_C - Z_A)\\
                                &\implies Z_I - Z_A = 2Z_C - 2Z_A\\
                                &\implies Z_I = 2Z_C - Z_A\\
                                &\implies Z_I = 2 (1 - i) - (2 - i)\\
                                &\implies Z_I = 2 - 2i - 2 + i\\
                                &\implies Z_I = -i.
\end{align*}

\[
\boxed{Z_I = -i}
\]


On a: $ J = \text{bar}\{(A,1); (C,2)\} $

\begin{align*}
\text{Donc } J = \text{bar}\{(A,1); (C,2)\} &\implies \overrightarrow{AJ} = \frac{2}{1+2} \overrightarrow{AC}\\
															 &\implies Z_J - Z_A = \frac{2}{3} Z_C - \frac{2}{3} Z_A\\
                               &\implies Z_J = \frac{2}{3} Z_C + \frac{1}{3} Z_A\\
                               &\implies Z_J = \frac{2}{3} (1 - i) + \frac{1}{3} (2 - i)\\
                               &\implies Z_J = \frac{2}{3} + \frac{2}{3} (-i) + \frac{2}{3} - \frac{i}{3}\\
                               &\implies Z_J = \frac{4}{3} - i
\end{align*}

\[
\boxed{Z_J = \frac{4}{3} - i}
\]

\item Donner un module et un argument de $\frac{Z_B - Z_A}{Z_C - Z_A}$

\begin{align*}
\frac{Z_B - Z_A}{Z_C - Z_A} &= \frac{2 - (2 - i)}{1 - i - (2 - i)}\\
														&= \frac{i}{-1}\\
														&= i
\end{align*}

\[
\text{Donc } \boxed{\frac{Z_B - Z_A}{Z_C - Z_A} = i}
\]

\textbf{Module}

\[
\left| \frac{Z_B - Z_A}{Z_C - Z_A} \right| = |i| = 1
\]

\textbf{Un Argument}

\[
\arg\left( \frac{Z_B - Z_A}{Z_C - Z_A} \right) = \frac{\pi}{2}
\]

\item Nature de $ABC$

\[
\begin{cases}
\left| \frac{Z_B - Z_A}{Z_C - Z_A} \right| = 1\\
\arg\left( \frac{Z_B - Z_A}{Z_C - Z_A} \right) = \frac{\pi}{2} 
\end{cases}\implies
\begin{cases}
AB = AC\\
\left(\overrightarrow{AC},\overrightarrow{AB} \right) = \frac{\pi}{2}
\end{cases}
\]
%\left( \widehat{ACB}, \widehat{CAB} \right) = \frac{\pi}{2}
\[
\text{Donc } ABC \text{ est un triangle rectangle et isocèle en } A.
\]

\item Montrons que $A, B, C, D$ sont cocycliques

On a :

\[
BD = |Z_D - Z_B| = |1 - 2i| = 1
\]

\[
DC = |Z_C - Z_D| = |1 - i - 2i| = 1
\]
\end{enumerate}

\section*{\underline{Problème :} 10 points}

application du théorème des inégalité des accroissement finie
\end{document}
