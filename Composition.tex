\documentclass[12pt,a4paper]{article}
\usepackage{amsmath,amssymb,mathrsfs,tikz,times,pifont}
\usepackage{enumitem}
\newcommand\circitem[1]{%
\tikz[baseline=(char.base)]{
\node[circle,draw=gray, fill=red!55,
minimum size=1.2em,inner sep=0] (char) {#1};}}
\newcommand\boxitem[1]{%
\tikz[baseline=(char.base)]{
\node[fill=cyan,
minimum size=1.2em,inner sep=0] (char) {#1};}}
\setlist[enumerate,1]{label=\protect\circitem{\arabic*}}
\setlist[enumerate,2]{label=\protect\boxitem{\alph*}}
%%%::::::by chnini ameur :::::::%%%
\everymath{\displaystyle}
\usepackage[left=1cm,right=1cm,top=1cm,bottom=1.7cm]{geometry}
\usepackage{array,multirow}
\usepackage[most]{tcolorbox}
\usepackage{varwidth}
\tcbuselibrary{skins,hooks}
\usetikzlibrary{patterns}
%%%::::::by chnini ameur :::::::%%%
\newtcolorbox{exa}[2][]{enhanced,breakable,before skip=2mm,after skip=5mm,
colback=yellow!20!white,colframe=black!20!blue,boxrule=0.5mm,
attach boxed title to top left ={xshift=0.6cm,yshift*=1mm-\tcboxedtitleheight},
fonttitle=\bfseries,
title={#2},#1,
% varwidth boxed title*=-3cm,
boxed title style={frame code={
\path[fill=tcbcolback!30!black]
([yshift=-1mm,xshift=-1mm]frame.north west)
arc[start angle=0,end angle=180,radius=1mm]
([yshift=-1mm,xshift=1mm]frame.north east)
arc[start angle=180,end angle=0,radius=1mm];
\path[left color=tcbcolback!60!black,right color = tcbcolback!60!black,
middle color = tcbcolback!80!black]
([xshift=-2mm]frame.north west) -- ([xshift=2mm]frame.north east)
[rounded corners=1mm]-- ([xshift=1mm,yshift=-1mm]frame.north east)
-- (frame.south east) -- (frame.south west)
-- ([xshift=-1mm,yshift=-1mm]frame.north west)
[sharp corners]-- cycle;
},interior engine=empty,
},interior style={top color=yellow!5}}
%%%%%%%%%%%%%%%%%%%%%%%

\usepackage{fancyhdr}
\usepackage{eso-pic}         % Pour ajouter des éléments en arrière-plan
% Commande pour ajouter du texte en arrière-plan
\AddToShipoutPicture{
    \AtTextCenter{%
        \makebox[0pt]{\rotatebox{80}{\textcolor[gray]{0.7}{\fontsize{5cm}{5cm}\selectfont PGB}}}
    }
}
\usepackage{lastpage}
\fancyhf{}
\pagestyle{fancy}
\renewcommand{\footrulewidth}{1pt}
\renewcommand{\headrulewidth}{0pt}
\renewcommand{\footruleskip}{10pt}
\fancyfoot[R]{
\color{blue}\ding{45}\ \textbf{2025}
}
\fancyfoot[L]{
\color{blue}\ding{45}\ \textbf{Prof:M. BA}
}
\cfoot{\bf
\thepage /
\pageref{LastPage}}
\begin{document}
\renewcommand{\arraystretch}{1.5}
\renewcommand{\arrayrulewidth}{1.2pt}
\begin{tikzpicture}[overlay,remember picture]
\node[draw=blue,line width=1.2pt,fill=purple,text=blue,inner sep=3mm,rounded corners,pattern=dots]at ([yshift=-2.5cm]current page.north) {\begingroup\setlength{\fboxsep}{0pt}\colorbox{white}{\begin{tabular}{|*1{>{\centering \arraybackslash}p{0.28\textwidth}} |*2{>{\centering \arraybackslash}p{0.2\textwidth}|} *1{>{\centering \arraybackslash}p{0.19\textwidth}|} }
\hline
\multicolumn{3}{|c|}{$\diamond$$\diamond$$\diamond$\ \textbf{Lycée de Dindéfélo}\ $\diamond$$\diamond$$\diamond$ }& \textbf{A.S. : 2024/2025} \\ \hline
\textbf{Matière: Mathématiques}& \textbf{Niveau : T}\textbf{S2} &\textbf{Date: 30/01/2024} & \textbf{Durée : 4 heures} \\ \hline
\multicolumn{4}{|c|}{\parbox[c]{10cm}{\begin{center}
\textbf{{\Large\sffamily composition n$ ^{\circ} $ 1 Du 1$ ^\text{\bf er} $ Semestre}}
\end{center}}} \\ \hline
\end{tabular}}\endgroup};
\end{tikzpicture}
\vspace{3cm}

\section*{\underline{Exercice 1 :} $1,5$ points}


\textbf{1. Déterminons les racines cubiques de l'unité.}

Résolvons pour cela $z^3 = 1$, $z \in \mathbb{C}$ :

\[
z^3 = 1 \implies z^3 = e^{i(0+2k\pi)}, \quad k \in \mathbb{Z}.
\]

\[
z = e^{i\frac{(0+2k\pi)}{3}}, \quad k = 0, 1, 2.
\]

- Pour $k = 0$, $z = e^{i0} = 1$, donc $z_0 = 1$.

- Pour $k = 1$, $z = e^{i\frac{2\pi}{3}} = \cos\frac{2\pi}{3} + i\sin\frac{2\pi}{3} = -\frac{1}{2} + i\frac{\sqrt{3}}{2}$, donc $z_1 = -\frac{1}{2} + i\frac{\sqrt{3}}{2}$.

- Pour $k = 2$, $z = e^{i\frac{4\pi}{3}} = \cos\frac{4\pi}{3} + i\sin\frac{4\pi}{3} = -\frac{1}{2} - i\frac{\sqrt{3}}{2}$, donc $z_2 = -\frac{1}{2} - i\frac{\sqrt{3}}{2}$.

Ainsi, l'ensemble des solutions est :

\[
S_{\mathbb{C}} = \left\{ 1, -\frac{1}{2} + i\frac{\sqrt{3}}{2}, -\frac{1}{2} - i\frac{\sqrt{3}}{2} \right\}.
\]

\textbf{1. Interprétation}

\begin{itemize}
    \item $\text{arg}(z_B - z_A) = (\overrightarrow{OI}, \overrightarrow{AB}) \, [2\pi]$
    
    Géométriquement, $\text{arg}(z_B - z_A)$ est l'angle $\widehat{(\overrightarrow{OI}, \overrightarrow{AB})}$ formé par le vecteur $\overrightarrow{AB}$ avec l'axe réel du plan complexe $(O, \overrightarrow{OI})$.

    \item $\text{arg}\left(\frac{z_B - z_C}{z_B - z_A}\right) = (\overrightarrow{AB}, \overrightarrow{CD}) \, [2\pi]$
    
    Géométriquement, $\text{arg}\left(\frac{z_B - z_C}{z_B - z_A}\right)$ est l'angle $\widehat{(\overrightarrow{AB}, \overrightarrow{CD})}$ que forment les vecteurs $\overrightarrow{AB}$ et $\overrightarrow{CD}$.
\end{itemize}

\textbf{3. $f(x) = \sqrt{x}$, appliquons l'IAF}

Appliquons l'IAF (Inégalité des Accroissements Finis) sur $[l, l+1]$.

\textbf{TI-AF 1}

Si $f$ est continue sur $[a, b]$ et dérivable sur $]a, b[$, avec 
\[
\exists (m, M) \in \mathbb{R}, \forall x \in [a, b], \quad \text{tel que } m \leq f'(x) \leq M,
\]
alors :
\[
m(b-a) \leq f(b) - f(a) \leq M(b-a).
\]

\textbf{TI-AF 2}

Si $f$ est continue sur $[a, b]$ et dérivable sur $]a, b[$, avec 
\[
\exists m \in \mathbb{R}_{+}, \forall x \in [a, b], \quad \text{tel que } |f'(x)| \leq m,
\]
alors :
\[
|f(b) - f(a)| \leq m|b-a|.
\]

\newpage
\textbf{Applications} $t > 0$

\begin{itemize}
    \item $f$ est continue sur $[t, t+1]$.
    \item $f$ est dérivable sur $]t, t+1[$ et pour tout,
\[
    \forall x \in ]t, t+1[, \quad f'(x) = \frac{1}{2\sqrt{x}}.
\]
\end{itemize}

Comme $x \in ]t, t+1[$, alors $t < x < t+1$. Ainsi, 
\[
\frac{1}{2\sqrt{t+1}} < \frac{1}{2\sqrt{x}} < \frac{1}{2\sqrt{t}}.
\]

D'où :
\[
\frac{1}{2\sqrt{t+1}} < f'(x) < \frac{1}{2\sqrt{t}}.
\]

En appliquant l'inégalité des accroissements finis (IAF), nous avons :
\[
\frac{1}{2\sqrt{t+1}}(t+1-t) < f(t+1) - f(t) < \frac{1}{2\sqrt{t}}(t+1-t).
\]

Cela donne :
\[
\frac{1}{2\sqrt{t+1}} < f(t+1) - f(t) < \frac{1}{2\sqrt{t}}.
\]


\section*{\underline{Exercice 2 :} 4 points}



\section*{\underline{Problème :} 12 points}


\end{document}
