\documentclass[a4paper,12pt]{article}
\usepackage{graphicx}
\usepackage[a4paper, top=0cm, bottom=2cm, left=2cm, right=2cm]{geometry} % Ajuste les marges
\usepackage{xcolor} % Pour ajouter des couleurs
\usepackage{hyperref} % Pour avoir des références colorées si nécessaire
\usepackage{eso-pic}         % Pour ajouter des éléments en arrière-plan

\usepackage[french]{babel}
\usepackage[T1]{fontenc}
\usepackage{mathrsfs}
\usepackage{amsmath}
\usepackage{amsfonts}
\usepackage{amssymb}
\usepackage{tkz-tab}

\usepackage{tikz}
\usetikzlibrary{arrows, shapes.geometric, fit}
\newcounter{correction} % Compteur pour les questions

% Définir une nouvelle couleur
\definecolor{myorange}{RGB}{255,165,0}

% Définir la commande pour afficher une question numérotée
\newcommand{\question}{%
  \refstepcounter{correction}%
  \textbf{\textcolor{myorange}{Question \thecorrection (1 point) :}} \ignorespaces
}
% Commande pour ajouter du texte en arrière-plan
\AddToShipoutPicture{
    \AtTextCenter{%
        \makebox[0pt]{\rotatebox{80}{\textcolor[gray]{0.9}{\fontsize{10cm}{10cm}\selectfont Pathé Gobel BA}}}
    }
}
\begin{document}

% En-tête
\begin{center}
    \begin{tabular}{@{} p{5cm} p{5cm} p{5cm} @{}} % 3 colonnes avec largeurs fixées
        Lycée Dindéfélo & Évaluation 1 & 23 Octobre 2024 \\
    \end{tabular}
    \\[-0.01cm] % Ajuster l'espace vertical entre le tableau et la barre
    \hrule % Barre horizontale
\end{center}
\begin{center}
    \textbf{\Large Évaluation 1 : Asymptotes et branches infinies} \\[0.2cm]
    \textbf{\large Professeur : M. BA} \\[0.2cm]
    \textbf{Classe : Terminale S2} \\[0.2cm]
    \textbf{\small Durée : 5 minutes} \\[0.2cm]
    \textbf{\small Note : /5}
\end{center}

% Nom de l'élève
\textbf{\small Nom de l'élève :} \underline{\hspace{8cm}} \\[0.5cm]

% Introduction aux questions
Complétez les questions suivantes en vous aidant du cours. \\[0.3cm]


\question\\
On dit que la droite d'équation \( y = b \) est une asymptote horizontale à la courbe représentative de \( f \) si et seulement si :\\

\[
\lim_{x \to +\infty} f(x) = \underline{\hspace{2cm}}, \quad \lim_{x \to -\infty} f(x) = \underline{\hspace{2cm}}
\]

\question\\
Complétez la phrase suivante : La droite d'équation \( x = a \) est une asymptote verticale à la courbe représentative de \( f \) si \underline{\hspace{15cm}}\\[0.3cm]
\underline{\hspace{20cm}}\\[0.3cm]
\question\\
Soit \( f(x) = \frac{x-2}{x-3} \). \\
Déterminez les limites de \( f(x) \) en \( x \to +\infty \), \( x \to -\infty \) et \( x \to 3 \) :
\[
\lim_{x \to +\infty} f(x) = \underline{\hspace{2cm}}, \quad \lim_{x \to -\infty} f(x) = \underline{\hspace{2cm}}, \quad \lim_{x \to 3^-} f(x) = \underline{\hspace{2cm}}, \quad 
\]
\[\lim_{x \to 3^+} f(x) = \underline{\hspace{2cm}}.\]

\question\\
Montrez que la droite \( y = x + 1 \) est une asymptote oblique de la fonction \( f(x) = \frac{x^2 - 1}{x-1} \) en \( +\infty \). \\[0.2cm]
\underline{\hspace{20cm}}\\[0.3cm]
\underline{\hspace{20cm}}\\[0.3cm]
\underline{\hspace{20cm}}\\[0.3cm]
\underline{\hspace{20cm}}\\[0.3cm]
\underline{\hspace{20cm}}\\[0.3cm]
\question\\
Vrai ou faux : Si une fonction \( f \) admet une asymptote horizontale en \( x \to +\infty \), alors elle ne peut pas avoir d'asymptote verticale. Expliquez votre réponse.\\[1cm]
\underline{\hspace{20cm}}\\[0.3cm]
\underline{\hspace{20cm}}\\[0.3cm]
\underline{\hspace{20cm}}\\[0.3cm]
\underline{\hspace{20cm}}\\[0.2cm]
\end{document}