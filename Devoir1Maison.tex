\documentclass[12pt,a4paper]{article}
\usepackage{amsmath,amssymb,mathrsfs,tikz,times,pifont}
\usepackage{enumitem}
\newcommand\circitem[1]{%
\tikz[baseline=(char.base)]{
\node[circle,draw=gray, fill=red!55,
minimum size=1.2em,inner sep=0] (char) {#1};}}
\newcommand\boxitem[1]{%
\tikz[baseline=(char.base)]{
\node[fill=cyan,
minimum size=1.2em,inner sep=0] (char) {#1};}}
\setlist[enumerate,1]{label=\protect\circitem{\arabic*}}
\setlist[enumerate,2]{label=\protect\boxitem{\alph*}}
%%%::::::by chnini ameur :::::::%%%
\everymath{\displaystyle}
\usepackage[left=1cm,right=1cm,top=1cm,bottom=1.7cm]{geometry}
\usepackage{array,multirow}
\usepackage[most]{tcolorbox}
\usepackage{varwidth}
\tcbuselibrary{skins,hooks}
\usetikzlibrary{patterns}
%%%::::::by chnini ameur :::::::%%%
\newtcolorbox{exa}[2][]{enhanced,breakable,before skip=2mm,after skip=5mm,
colback=yellow!20!white,colframe=black!20!blue,boxrule=0.5mm,
attach boxed title to top left ={xshift=0.6cm,yshift*=1mm-\tcboxedtitleheight},
fonttitle=\bfseries,
title={#2},#1,
% varwidth boxed title*=-3cm,
boxed title style={frame code={
\path[fill=tcbcolback!30!black]
([yshift=-1mm,xshift=-1mm]frame.north west)
arc[start angle=0,end angle=180,radius=1mm]
([yshift=-1mm,xshift=1mm]frame.north east)
arc[start angle=180,end angle=0,radius=1mm];
\path[left color=tcbcolback!60!black,right color = tcbcolback!60!black,
middle color = tcbcolback!80!black]
([xshift=-2mm]frame.north west) -- ([xshift=2mm]frame.north east)
[rounded corners=1mm]-- ([xshift=1mm,yshift=-1mm]frame.north east)
-- (frame.south east) -- (frame.south west)
-- ([xshift=-1mm,yshift=-1mm]frame.north west)
[sharp corners]-- cycle;
},interior engine=empty,
},interior style={top color=yellow!5}}
%%%%%%%%%%%%%%%%%%%%%%%

\usepackage{fancyhdr}
\usepackage{eso-pic}         % Pour ajouter des éléments en arrière-plan
% Commande pour ajouter du texte en arrière-plan
\AddToShipoutPicture{
    \AtTextCenter{%
        \makebox[0pt]{\rotatebox{80}{\textcolor[gray]{0.7}{\fontsize{5cm}{5cm}\selectfont PGB}}}
    }
}
\usepackage{lastpage}
\fancyhf{}
\pagestyle{fancy}
\renewcommand{\footrulewidth}{1pt}
\renewcommand{\headrulewidth}{0pt}
\renewcommand{\footruleskip}{10pt}
\fancyfoot[R]{
\color{blue}\ding{45}\ \textbf{2025}
}
\fancyfoot[L]{
\color{blue}\ding{45}\ \textbf{Prof:M. BA}
}
\cfoot{\bf
\thepage /
\pageref{LastPage}}
\begin{document}
\renewcommand{\arraystretch}{1.5}
\renewcommand{\arrayrulewidth}{1.2pt}
\begin{tikzpicture}[overlay,remember picture]
\node[draw=blue,line width=1.2pt,fill=purple,text=blue,inner sep=3mm,rounded corners,pattern=dots]at ([yshift=-2.5cm]current page.north) {\begingroup\setlength{\fboxsep}{0pt}\colorbox{white}{\begin{tabular}{|*1{>{\centering \arraybackslash}p{0.28\textwidth}} |*2{>{\centering \arraybackslash}p{0.2\textwidth}|} *1{>{\centering \arraybackslash}p{0.19\textwidth}|} }
\hline
\multicolumn{3}{|c|}{$\diamond$$\diamond$$\diamond$\ \textbf{Lycée de Dindéfélo}\ $\diamond$$\diamond$$\diamond$ }& \textbf{A.S. : 2024/2025} \\ \hline
\textbf{ }& \textbf{Niveau : T}\textbf{S2} &\textbf{Date: 03/01/2025} &  \\ \hline
\multicolumn{4}{|c|}{\parbox[c]{10cm}{\begin{center}
\textbf{{\Large\sffamily homework n$ ^{\circ} $ 1 Du 1$ ^\text{\bf er} $ Semestre}}
\end{center}}} \\ \hline
\end{tabular}}\endgroup};
\end{tikzpicture}
\vspace{3cm}

\section*{\underline{Exercice 1 :} }
\begin{enumerate}
    \item Déterminer le module et un argument des nombres complexes suivants :
    \begin{enumerate}
        \item $2i$ ;
        \item $\sqrt{3} + 3i$ ;
        \item $\sqrt{6} + i\sqrt{2}$ ;
        \item $5$ ;
        \item $\left(\frac{1}{2}-i\frac{\sqrt{3}}{2}\right)^3$ ;
        \item $\left(\frac{1}{2}+i\frac{\sqrt{3}}{2}\right)\left(-\frac{1}{2}+i\frac{\sqrt{3}}{2}\right)$ ;
        \item $\frac{-1+i\sqrt{3}}{\sqrt{3} + i}$.
    \end{enumerate}
    \item Écrire sous forme trigonométrique les nombres complexes suivants :
    \begin{enumerate}
        \item $(2+2i)(1-i)$ ;
        \item $\frac{-1+i\sqrt{3}}{1+i}$ ;
        \item $\frac{\sqrt{2}}{1+i}$ ;
        \item $\frac{-2i}{1+i\sqrt{3}}$ ;
        \item $(-1-i)^4$ ;
        \item $\left(\frac{1+i\sqrt{3}}{1-i}\right)^2$.
    \end{enumerate}
    \item Soit $z_1 = 1+i$ et $z_2 = 1+i\sqrt{3}$.
    \begin{enumerate}
        \item Déterminer le module et un argument de $z_1$ et $z_2$.
        \item Écrire sous forme algébrique et sous forme trigonométrique le produit $z_1z_2$.
        \item En déduire les valeurs de $\cos\frac{7\pi}{12}$ et $\sin\frac{7\pi}{12}$.
    \end{enumerate}
\end{enumerate}

\end{document}
